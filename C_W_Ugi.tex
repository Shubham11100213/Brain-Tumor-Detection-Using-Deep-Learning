\documentclass{article}
\usepackage[utf8]{inputenc}
\usepackage{graphicx}
\usepackage[top=1in, bottom=1in, left=1in, right=1in]{geometry} 
\date{}
\author{}
\begin{document}
\begin{titlepage}
    \begin{center}
    
        \begin{figure}[h]
            \centering
            \includegraphics[width=80mm]{a.jpg}
        \end{figure}
          \vspace{5mm}
        \textsc{\large\textbf{UNITED INSTITUTE OF TECHNOLOGY PRAYAGRAJ}\\
        Uttar Pradesh 211003, INDIA.\\
        2022-2023}\\
        \vspace{5mm}
        \textsc{\large\textbf{Department of Computer Science & Engineering}}\\
        
        \vspace{10mm}
        \LARGE A \\[0.5cm] Project Synopsis\\[0.5cm] on\\[0.5cm]
        {\LARGE  \bfseries Brain Tumor Detection Using Deep Learning }\\[0.5cm]
        
         {\large
        Under The Guidance Of\\[0.2cm]
        \textbf{Mr.Praful Pandey } \\[0.2cm]
        (Assistant Professor)\\[0.5cm]
        } 

        {\large 
        Submitted By-\\[0.3cm]
        \\
        1. Shubham Maurya (1902840100094)\\
2. Sudhanshu Maurya (1902840100099) 
        \\[0.1cm]
        }
        \\[0.5]
        

    \end{center}
\end{titlepage}
\maketitle
\Large
\tableofcontents

\newpage



\section{Introduction}

\subsection{Purpose}
\thispagestyle{empty}

\doublespace

\qquad 

The human body is made up of many organs and brain is the most critical and vital organ of them all.
One of the common reasons for dysfunction of brain is brain tumor. A tumor is nothing but excess cells
growing in an uncontrolled manner. Brain tumor cells grow in a way that they eventually take up all the
nutrients meant for the healthy cells and tissues, which results in brain failure. Currently, doctors locate
the position and the area of brain tumor by looking at the MR Images of the brain of the patient manually.
This results in inaccurate detection of the tumor and is considered very time consuming. A Brain Cancer
is very critical disease which causes deaths of many individuals. The brain tumor detection and
classification system is available so that it can be diagnosed at early stages. Cancer classification is the
most challenging tasks in clinical diagnosis. This project deals with such a system, which uses computer,
based procedures to detect tumor blocks and classify the type of tumor using Convolution Neural Network
Algorithm for MRI images of different patients. Different types of image processing techniques like
image segmentation, image enhancement and feature extraction are used for the brain tumor detection in
the MRI images of the cancer-affected patients. Detecting Brain tumor using Image Processing techniques
its involves the four stages is Image Pre-Processing, Image segmentation, Feature Extraction, and
Classification. Image processing and neural network techniques are used for improve the performance of
detecting and classifying brain tumor in MRI images\\\\\\
OVERVIEW OF BRAIN AND BRAIN TUMOR\\ Main part in human nervous system is human brain. It
is located in human head and it is covered by the skull. The function of human brain is to control all the
parts of human body. It is one kind of organ that allows human to accept and endure all type of
environmental condition. The human brain enables humans to do the action and share the thoughts and
feeling. In this section we describe the structure of the brain for understanding the basic things\\\\\\\\The brain tumors are classified into mainly two types: Primary brain tumor (benign tumor) and secondary
brain tumor (malignant tumor).The benign tumor is one type of cell grows slowly in the brain and type
of brain tumor is gliomas. It originates from non neuronal brain cells called astrocytes. Basically primary
tumors are less aggressive but these tumors have much pressure on the brain and because of that, brain
stops working properly . The secondary tumors are more aggressive and more quick to spread into other
tissue. Secondary brain tumor originates through other part of the body. These type of tumor have a cancer
cell in the body that is metastatic which spread into different areas of the body like brain, lungs etc.
Secondary brain tumor is very malignant. The reason of secondary brain tumor cause is mainly due to 
}
\newpage
\textbf{MAGNETIC RESONANCE IMAGING (MRI)}\\\\\\\\

Raymond v. Damadian invented the first magnetic image in 1969. In 1977 the first MRI image were
invented for human body and the most perfect technique. Because of MRI we are able to visualize the
details of internal structure of brain and from that we can observe the different types of tissues of human
body. MRI images have a better quality as compared to other medical imaging techniques like X-ray and
computer tomography.[8]. MRI is good technique for knowing the brain tumor in human body. There are
different images of MRI for mapping tumor induced Change including T1 weighted, T2 weighted and
FLAIR (Fluid attenuated inversion recovery) weighted shown in figure
\begin{figure}[h]
            \centering
            \includegraphics[width=170mm]{B.png}
        \end{figure}
 
 The most common MRI sequence is T1 weighted and T2 weighted. In T1 weighted only one tissue type is
bright FAT and in T2 weighted two tissue types are Bright FAT and Water both. In T1 weighted the
repetition time (TR) is short in T2 weighted the TE and TR is long. The TE an TR are the pulse sequence
parameter and stand for repetition time and time to echo and it can be measured in millisecond(ms)[9]. The
echo time represented time from the centre of the RF pulse to the centre of the echo and TR is the length of
time between the TE repeating series of pulse and echo is shown in figure 
\newpage
\begin{figure}[h]
            \centering
            \includegraphics[width=170mm]{c.png}
        \end{figure}
        The third commonly used sequence in the FLAIR. The Flair sequence is almost same as T2-weighted
image. The only difference is TE and TR time are very long. Their approximate TR and TE times are
shown in table. 
 \begin{figure}[h]
            \centering
            \includegraphics[width=170mm]{d.png}
        \end{figure}
\newpage
\subsubsection{ Design Model}
\begin{itemize}
    \item Agile Model
     \begin{figure}[h]
            \centering
            \includegraphics[width=170mm]{e.png}
        \end{figure}
\end{itemize}
\textbf{1. Requirements gathering and analysis}\\
In this phase, you must define the requirements. You should explain business opportunities and
plan the time and effort needed to build the project. Based on this information, you can evaluate
technical and economic feasibility.\\
\textbf{2. Design the requirements}\\
When you have identified the project, work with stakeholders to define requirements. You can
use the user flow diagram or the high-level UML diagram to show the work of new features and
show how it will apply to your existing system.\\
\textbf{3. Construction/ Iteration}\\
When the team defines the requirements, the work begins. The designers and developers start
working on their project. The aims of designers and developers deploy the working product within the estimated time. The product will go into various stages of improvement, so it includes
simple, minimal functionality.\\
\textbf{4.Deployment }\\
In this phase, the team issues a product for the user's work environment.\\
\textbf{5. Testing}\\
In this phase, the Quality Assurance team examine the product's performance and look for the
bug.\\
\textbf{6. Feedback}\\
After releasing of the product, the last step is to feedback it. In this step, the team receives
feedback about the produOverall Descriptions ct and works through the feedback. \\
\newpage


\susubsection{\textbf{Data Flow Diagram}}
\begin{figure}[h]
            \centering
            \includegraphics[width=170mm]{f.png}
        \end{figure}
\subsection{Document Conventions}
This document was based on the IEEE template for System Requirement Specifications Documentations
\subsection{Intended audience Suggestions  }
\item Audience of this SRS are other projects developer, users like student, viewers that will use System.
This SRS contain detail description about the product, its Functionality, different external interface
required , System features, Nonfunctional requirements and some additional requirements. 

\subsection{Product scope}
The main motivation behind Brain tumor detection is to not only detect tumor but it can also classify types
of tumor. So it can be useful in cases such as we have to sure the tumor is positive or negative, it can detect
tumor from image and return the result tumor is positive or not. This project deals with such a system,
which uses computer, based procedures to detect tumor blocks and classify the type of tumor using
Convolution Neural Network Algorithm for MRI images of different patients. 
\newpage
\section{\textbf{Overall Descriptions}}

\subsection{ Product Perspectives}
The main aim is to design a system is to predict whether the person is suffering from the brain tumour or not In
this process it collect the various type of images which contain the T1 weight images, T2 weight images and Flair
images and then pass through the Convolution Neural network (CNN) for the Segmentation and feature extraction
and Data Augumentation for training the data set. 

\subsection{ Product Functions }
\item Collect the images from the webscraping, api, images datasets from the Kaggle.
\item It includes the Data Cleaning and Data preprocessing which include the process of TF dataset(Tensorflow
dataset pipelining) and the Data Augumentation, Morphological operation, Feature Extraction
\item we are having dataset previously collected brain MRIs from which we are extracting features. 

\subsection{User Classes and characteristics}
Identify the various user classes that you anticipate will use this product. User classes may be differentiated
based on the frequency of use, a subset of product functions used, technical expertise, security or privilege
levels, educational level, or experience. Describe the pertinent characteristics of each user class. Certain
requirements may pertain only to certain user classes. Distinguish the most important user classes for this
product from those who are less important to satisfy.
\item User: User can capture or upload image and view result.
\item Admin: These user has an authority to update , delete and train sample image(training dataset).
\subsection{Operating environment }
Our Project is based Cloud and user application run on any web browser and plateform. So, we need desktop or
android smartphone to run an application.
\item Window 10
\item Window 11
\item Linux 
\subsection{Design and Implementations Constraints}
This project is developed in Python, it runs on Google collabs and Jupyter Notebook. It is wrapped into separate
modules where each module and each module are linked with each other, where we create various modules of
python which include the tensorflow , tf server , and many Model optimization technique.
\subsection{User Documentations }
The Software is uploaded on the open-source platform on google Collaboratory
\subsection{Assumptions and Dependencies}
Leaf disease detection is developed in Python. Python must be installed on the user System. The Jupyter Notebook
installed all the machine Learning Libraries and the Deep learning libraries. 
\section{\textbf{Project Requirements}}

\subsection{User Interface Requiremnets }


    \textbf{1.Images collection as Tensorflow TF}\\
In this process the cateogorical images of the leaf which include the healthy images , T1 weight
images, T2 images can be inserted from the local memory to the tensorflow modules in the form of
batches . in this process we use the batch processing technique in which the data is being taken from
the local memory in the batches. \begin{figure}[h]
            \centering
            \includegraphics[width=170mm]{g.png}
        \end{figure}
    
                        \begin{figure}[h]
            \centering
            \includegraphics[width=170mm]{h.png}
        \end{figure}
        \newpage
        \textbf{2.Preprocessing}\\
        Images come in different shapes and sizes. They also come through different sources. For example,
some images are what we call “natural images”, which means they are taken in color, in the real world.
For example:
\item A picture of a flower is a natural image.
\item An X-ray image is not a natural image.
Taking all these variations into consideration, we need to perform some pre-processing on any image
data. RGB is the most popular encoding format, and most “natural images” we encounter are in RGB.
Also, among the first step of data pre-processing is to make the images of the same size. Let’s move
on to how we can change the shape and form of images
\textbf{Morphological Transformations}\\
The term morphological transformation refers to any modification involving the shape and form of
the images. These are very often used in image analysis tasks. Although they are used with all types
of images, they are especially powerful for images that are not natural (come from a source other than
a picture of the real world).\\\\
1.Thresholding\\
one of the simpler operations where we take all the pixels whose intensities are above a certain
threshold and convert them to ones; the pixels having value less than the threshold are converted to
zero. This results in a binary image.\\\\
2.Erosion, Dilation, Opening & Closing\\
Erosion shrinks bright regions and enlarges dark regions. Dilation on the other hand is exact opposite
side — it shrinks dark regions and enlarges the bright regions.\\\\
Opening is erosion followed by dilation. Opening can remove small bright spots (i.e. “salt”) and
connect small dark cracks. This tends to “open” up (dark) gaps between (bright) features.\\\\
Closing is dilation followed by erosion. Closing can remove small dark spots (i.e. “pepper”) and
connect small bright cracks. This tends to “close” up (dark) gaps between (bright) features.\\\\
All these can be done using the skimage.morphology module. The basic idea is to have a circular disk
of a certain size (3 below) move around the image and apply these transformations using it.\\\\
\newpage
\textbf{3.Segmentation}\\\\
Segmentation: Region growing is the simple region-based image segmentation technique. It is also
classified as a pixel based image segmentation technique since it is involve the selection of initial
seed points.
\begin{figure}[h]
            \centering
            \includegraphics[width=170mm]{i.png}
        \end{figure}\\\\
        \textbf{4.Feature Extraction}\\\\We apply machine learning to analyze a growing volume of data, which is becoming
increasingly complicated. The emergence of deep learning during the past decade has undoubtedly
helped generate learning paradigms that are becoming more effective. Numerous machine-learning
jobs seek to categorize difficulties. Since features are extracted from the input data, it can be
considered as a novel representation of the data, particularly for this task. Subsequently, in addition
to these features, which complete the job, a categorization method is learned. This method should be
applied to unobserved data in the training stage period upon completion of the training. It ought to
give a precise forecast of its response precisely and in this situation the class label. Frequently, and
particularly until recent times, the features extracted from the input were handcrafted, implying that
they are specially designed for the input data and the present task. It is standard practice for these not
to be exclusively tied to the data type; for instance, handwritten images of words' pictures, but rather
to a specific subset, such as English words handwritten in ink on parchment. Usually, most such
features cannot manage change well; nevertheless, machine learning is a different method of
extracting features from the data to learn a feature extractor.\\\\\ 
\textbf{Convolutional Neural Network (CNN)}\\
convolutional layers store the output of the kernels from the previous layer which consists of
weights and biases to be learned. The generated kernels that represent the data without an error is
the point of the optimization function. In this layer, a sequence of mathematical processes is done
to extract the feature map of the input image .exhibits the operation of the convolution layer for a
5x5 image input and aresult is a 3x3 filter that reduced to a smaller size . Also, the figure shows
the shifting of filter starting from the upper left corner of the input image. The valuesfor each step
are thenmultiplied by the values of the filter and the added values are the result. A new matrix
with the reduced size is formed from the input image.\\
\textbf{5. Dimension Reduction}\\
 The availability of high-dimensional medical image data during the identification procedure can place
a heavy computational burden and require a suitable preprocessing step for lower-dimensional
representation. At the same time, to reduce the storage requirement and complexity of the image data
Random Projection Technique (RPT) is widely accepted as the multivariate approach for data
reduction. Aims This paper mainly focuses on T1-weighted MRI images clustering for brain tumor
segmentation with dimension reduction by using a conventional Principle Component Analysis (PCA)
and RPT. Methods Two clustering algorithms, K-Means and Fuzzy C-Means, are used to detect the
brain tumor. The primary objective is to present a comparison of two cluster methods between the
PCA algorithm and RPT on MRIs. Apart from the original dimension of 512×512, the analysis used
three other sizes, 256×256, 128×128, and 64×64, to study the effect of methods.\\\\
\textbf{6. Classification}\\
Classification is the final step of the image analysis method that involves sorting feature data in an
image into separate classes. After segmenting a suspicious region, feature extraction and selection
scheme are performed to extract the relevant information from the region; and a classification
technique is used so that the best results are achieved, based on the available features and the Tumor
classes.\\\\ \begin{figure}[h]
            \centering
            \includegraphics[width=170mm]{j.png}
        \end{figure}\\\\
       \textbf{ 7. Performance Analysis}\\
It needed to look into the most recent cutting-edge studies on brain tumor identification and tracking.
This study evaluated recent papers published in the last decade or so that focused on the identification
and categorization/classification of brain tumors employing CNN.\\\
Current framework systems follow multiple pre-defined procedures to identify brain MRI images The
effective mechanisms involved in recognizing and classifying tumor and non-tumor units in MRI
brain imaging is covered in . The following is a brief overview of prospective approaches and
strategies. Most images to be entered as input are MRI brain scans . Depending on the architecture
and memory limitations, the input might be 2D or 3D. Due to its efficiency in significantly enhancing
image data the input regarding images to be entered has proved to be as crucial as any other stage .
Segmentation primarily divides the input image into identical sections depending upon specified
criteria, allowing only essential data to be extracted and the remainder to be discarded . There are
numerous approaches. Some studies segment the actual tumor [48], whereas others segment the image
region including the tumor . The goal of the classification stage is to divide the input data into various
categories based on comparable behavior patterns inside the group. The third process described in the
literature involves directly feeding brain MRI images to a Deep Learning program for categorization
with no pre-processing. Statistical techniques or machine learning techniques are used to identify
features. The Deep Learning algorithm is then trained using these extracted features. While deep
learning methods do Appl. Sci. 2022, 12, 7282 8 of 20 not necessitate extraction of features, the study
has shown that extracted features using machine learning, or meta-heuristic optimization methods, are
still used in different models with reinforcement learning to include efficient and resilient features .
Each plan’s principal purpose is to change the levels of Supervised Learning based on the
experimental criteria and then choose the model with the best performance. Using machine and Deep
Learning approaches, researchers employed models to build efficient systems. The dataset is divided
into learning, testing, and verification sets before beginning any of the approaches above.
Convolutional Neural Network (CNN) has received substantial appreciation and recognition in Deep
Learning for its ability to automatically extract and detect deep features by responding to tiny changes
in images. 

\subsection{Hardware Interface }
The minimum hardware requirement of the Application is a 500 Megahertz CPU and 500 megabytes of RAM.
Also, because it uses the Deep Learning Concept where we use the computational power so we need a Compatible
graphic card at least NVIDIA 1650. for the bigger Networks, additional memory is required
\subsection{Software Interface}
The application required python and Scikit-learn modules to be installed on the System, more specifically Python
3.6 version.\\\\
The project can be connected with the Deep Learning APIs and the API for fetching the real time Photo. 
\subsection{Communication Interface}
The project requires an internet connection to install new plugins, update already installed ones, and update some
of its Components(APIs, modules, etc). 
\section{\textbf{System Features}}
This section demonstrates the most prominent part of this Software where it tells us about the accuracy of the
model that how much accuracy it performs for the training model and the testing model.\\
\subsection{\textbf{System feature}}\\\\
\large
4.1.1.Description and priority\\
 Detect The MRI whether the person has Tumor or Not.\\\\
4.1.2.Stimulus/Response Sequences
\item Give quick response when leaf capture.
\item  Run system in using previously fetched dataset when not connected to cloud (Internet). \\




\newpage

\section{References}
\item 1.L.Guo,L.Zhao,Y.Wu,Y.Li,G.Xu,andQ.Yan,“Tumordetection in MR images using oneclass immune feature weighted
SVMs,” IEEE Transactions on Magnetics, vol. 47, no. 10, pp. 3849–3852,2011.\\
\item 2.R.Kumari,“SVMclassificationanapproachondetectingabnormalityinbrainMRIimages,”Inter
nationalJournalofEngineeringResearchandApplications,vol.3,pp.1686–1690,2013.\\
\item 3.DICOM Samples Image Sets, http://www.osirix-viewer.com/.\\
\item 4.“Brainweb:SimulatedBrainDatabase” http://brainweb.bic.mni.mcgill.ca/cgi/brainweb1.\\
\item 5.Obtainable Online: www.cancer.ca/~/media/CCE 10/08/2015.\\
\item 6. J. C. Buckner, P. D. Brown, B. P. O’Neill, F. B. Meyer , C. J. Wetmore,J. H Uhm, "Central nervous system tumors." In
Mayo Clinic Proceedings,Vol. 82, No. 10, pp. 1271- 1286, October 2007.\\
\item 7.Deepa , Singh Akansha. (2016). - Review of Brain Tumor Detection from tomography. International Conference on
Computing for Sustainable Global Development (INDIACom) 

\end{document}



